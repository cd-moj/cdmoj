\section{Problema H: Desfile dos Patos}
\vspace{-0.52cm}
\noindent \begin{verbatim}Arquivo: desfile.[c|cpp|pas]\end{verbatim}

Em uma pacata cidade do interior um curioso desfile acontece toda manhã às seis
horas. O desfile dos Patos acontece na avenida mais badalada da cidade (a Av.
Tupi). Esse desfile é tão reconhecido que pelo menos uma vez por semana
a televisão local filma o evento e transmite para a micro-região. Os patos
sempre saem para seu desfile alimentados e percorrem a avenida como verdadeiros
reis da cidade. Não é por acaso que a cidade possui o Trevo do Patinho com
a estátua do mais reconhecido Pato que viveu nessa cidade, o famoso
Pato Branco de polainas.

Durante o desfile dessa manhã, Bozena, percebeu que vários Patos possuem
uma mecha em suas penas. Essas mechas são um filete de alguma cor. Marciano,
um aluno de uma escola local, percebeu que uma cor é a majoritária (mais da
metade dos patos tem essa cor) no conjunto de todas as cores nas mechas, porém,
Patrick (colega de Marciano) não consegue decidir qual é a cor majoritária,
algumas cores parecem ter a maioria por pouca diferença e por isso é difícil
saber qual é a majoritária. Então Patrick o desafiou a escrever um programa de
computador que dada uma sequência das cores que aparecem nos patos durante o
desfile diga qual é a cor majoritária.

\subsection*{Entrada}

A entrada possui vários casos de teste. A primeira linha de um caso de teste
possui um número $N$ ($1 \leq N \leq 5000$ )que representa quantos patos foram observados. A segunda
linha de um caso de teste possui $N$ inteiros, $a_i$ ( $1 \leq a_i \leq 10^5$),
separados por um espaço em branco, correspondendo ao código da cor que estava
na mecha do Pato. A entrada termina quando $N = 0$.

\subsection*{Saída}

Para cada caso de teste imprima, em uma única linha, o código da cor que é
a majoritária no desfile.

\subsection*{Exemplo}

\begin{multicols}{2}
\subsubsection*{Entrada}
\begin{verbatim}
5
1 0 1 2 1
13
1 1 1 3 3 2 2 3 3 3 2 3 3
\end{verbatim}
\columnbreak
\subsubsection*{Saída}
\begin{verbatim}
1
3
\end{verbatim}
\end{multicols}
