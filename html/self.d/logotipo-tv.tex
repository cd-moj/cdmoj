\section{Problema F: Logotipo publicitário na TV da Vovó}
\vspace{-0.52cm}
\noindent \begin{verbatim}Arquivo: logotipo.[c|cpp|pas]\end{verbatim}

A Vové está preocupada com seu netinho que passa o dia inteiro assistindo televisão, pois ela percebeu que durante o desenho animado, aparece um logotipo publicitário na tela e ela não está muito contente com isso.

O netinho e a Vovó gravaram em seu videocassete alguns programas e agora desejam contar quantas vezes o logotipo aparece durante o desenho animado para decidir se reclamam com a emissora por propaganda abusiva.

A Vovó não quer assistir ao vídeo mais uma vez para fazer a contagem e tão pouco deseja que seu netinho o faça. Por isso ela contratou você para descobrir quantas vezes o logotipo aparece em cada vídeo gravado.

\subsection*{Tarefa}

Dada uma matriz representando o logotipo que a Vovó deseja contar, e dadas K matrizes, cada uma representando uma imagem do vídeo gravado, seu programa deve contar quantas vezes a matriz do logotipo aparece na sequência de K matrizes. A matriz do logotipo é sempre de dimensões menores que as K matrizes do vídeo.

\subsection*{Entrada}
A entrada possui vários casos de teste. Cada caso de teste inicia com a descrição da matriz que representa o logotipo, seguida de K matrizes, cada uma representando uma das imagens do vídeo que a Vovó gravou.

A primeira linha contém dois inteiros $X$ e $Y$ representando respectivamente o número de linhas e colunas da matriz do logotipo ($1 \leq X \leq 10$ e $1 \leq Y \leq 10$). As X linhas seguintes da entrada contém cada uma Y inteiros, descrevendo o valor de cada ponto da matriz do logotipo.

Após a descricão do logotipo, são descritas as matrizes que representam as diversas imagens do vídeo gravado. A primeira linha da descrição possui três inteiros $K$, $M$ e $N$. Onde K representa o número de matrizes de dimensão $M \times N$ que serão lidas ( $1 \leq K \leq 300$, $1 \leq M \leq 320$, $1 \leq N  \leq 240$). Cada pixel é um número $p_i$ com $0 \leq p_i \leq 255$.
O final da entrada é indicado por $X = Y = 0$.

\subsection*{Saída}

Para cada caso de teste, o seu programa deve produzir um número na saída. A primeira linha da saída deve conter um identificador do conjunto de teste, no formato “Logotipo n”, onde $n$ é numerado sequencialmente a partir de $1$. A seguir deve aparecer um número representando quantas vezes o logotipo aparece no vídeo gravado. Após o número deixe uma linha em branco.

\subsection*{Exemplo de Entrada}
(próxima página)

\newpage
\begin{multicols}{2}
\subsubsection*{Entrada}
\begin{verbatim}
2 2
1 1
2 3
3 5 5
0 0 0 2 3
0 1 1 0 0
1 3 3 0 1
3 0 0 0 2
0 0 0 1 1
0 3 0 0 0
0 0 0 1 1
0 9 0 2 3
2 3 0 0 0
1 1 8 0 0
0 1 1 1 0
0 1 0 1 0
0 1 0 1 1
0 1 0 1 3
0 1 1 1 0
1 2
9 9
2 4 4
2 9 9 2
3 3 3 8
8 7 9 9
9 9 2 9
2 6 1 3
9 2 8 9
0 3 4 0
0 0 9 9
0 0
\end{verbatim}
\columnbreak
\subsubsection*{Saída}
\begin{verbatim}
Logotipo 1
1

Logotipo 2
4

\end{verbatim}
\end{multicols}
